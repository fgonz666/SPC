\documentclass[12pt, a4paper]{article}
\usepackage[utf8]{inputenc}
\usepackage[T1]{fontenc}
\usepackage[french]{babel}
%\usepackage[left=20mm, right=20mm, top=20mm, bottom=20mm]{geometry}
\usepackage[table]{xcolor}
\usepackage{amsmath}
\usepackage{amsfonts}
\usepackage{amssymb}
\usepackage{lmodern}
\usepackage{tikz}
\usepackage{hyperref}
	\hypersetup{
		colorlinks=true,
		urlcolor=blue!75!green,
		linkcolor=green!80!blue,
	}
	\urlstyle{same}
\usepackage{float}
\usepackage{array}
\usepackage[european]{circuitikz}
	\usetikzlibrary{math, babel}
\title{Exercices d'entraînement\\pour l'intensité\\du courant électrique.}
\author{F.S.G.}
\date{Document créé le 9 décembre 2021\\Compilation du \today{}.}
\renewcommand{\baselinestretch}{1.25}
\setlength{\parskip}{0.5em}
\begin{document}
\maketitle

\section*{Exercice 1}
\paragraph{Données : } à la sortie de la D.E.L. l'intensité vaut I~=~40~mA.

\begin{figure}[H]
	\centering
	\begin{circuitikz}
		\draw (0,3) to[vsource, voltage=american, v=9V, *-*] ++(0,-2) ;
		\draw (0,3) |- ++(1,1) to[leD-, *-*] ++(2,0) -| ++(1,-1) to[lamp, *-*] ++(0,-2) |- ++(-1,-1) to[R, *-*] ++(-2,0) -| ++(-1,1) ; 
	\end{circuitikz}
\end{figure}

\paragraph{1.} Recopiez le circuit sur votre feuille et ajoutez le courant électrique sur chaque fil avec à côté de chaque flèche l'intensité du courant correspondante.

Vous trouverez la localisation de chaque intensité dans le tableau.

\begin{table}[H]
	\centering
	\rowcolors{1}{white}{white!80!black}
	\renewcommand*{\baselinestretch}{1}
	\begin{tabular}{|m{0.18\linewidth}||m{0.20\linewidth}|m{0.16\linewidth}|m{0.16\linewidth}|m{0.20\linewidth}|}
		\hline
		Nom de l'intensité & $I_1$ & $I_2$ & $I_3$ & $I_4$ \\
		Localisation : Entre ... & {\small ...le pôle positif du générateur et la D.E.L.} & {\small ... la D.E.L. et la lampe} & {\small ... la lampe et le résistor} & {\small ... le résistor et le pôle négatif du générateur} \\
		Valeur mesurée & & 40~mA & & \\
		\hline
	\end{tabular}
\end{table}

\paragraph{3.} Donnez le nom de la loi sur l'intensité du courant qui doit être utilisée (appliquée) dans ce circuit et énoncez-là.

\paragraph{4.} En utilisant la loi de la 3\ieme{} question, complétez les cases vides du tableau.

\section*{Exercice 2}
Convertir les valeurs de la première unité vers la seconde.

\textbf{A.} 2~A $\rightarrow$ mA \hfill \textbf{B.} 56~mA $\rightarrow$ A \hfill \textbf{C.} 12~A $\rightarrow$ kA

\textbf{D.} 0,047~A $\rightarrow$ mA \hfill \textbf{E.} 3 $\mu$A $\rightarrow$ mA \hfill \textbf{F.} 7~A $\rightarrow$ $\mu$A

\paragraph{Aide :} Tableau des unités.
\begin{table}[H]
	\centering
	\begin{tabular}{|c|c|c|c|c|c|c|c|c|c|}
		\hline
		kA & ~~ & ~~ & ~A & ~~ & ~~ & mA & ~~ & ~~ & $\mu$A \\
		\hline
		& & & & & & & & & \\
		& & & & & & & & & \\
		& & & & & & & & & \\
		\hline
	\end{tabular}
\end{table}

\section*{Exercice 3}
\paragraph{Données :} l'ampèremètre $A_1$ mesure une intensité de 16,2~mA.

\begin{figure}[H]
	\centering
	\begin{circuitikz}
		\draw (3,1) to[vsource, *-*, voltage=american, v=9V] ++(2,0) ;
		\draw (3,1) -| ++(-1,-1) to[rmeter, t=$A_1$, *-*] ++(0,-2) |- ++(1,-1) to[R, *-*] ++(2,0) to[rmeter, t=$A_2$,*-*] ++(2,0) -| ++(1,1) to[leD-,*-*,mirror] ++(0,2) |- ++(-3,1) ;
	\end{circuitikz}
\end{figure}

\paragraph{a.} Redessinez le circuit et ajoutez les courants électriques près de chaque ampèremètre.

\paragraph{b.} Choisissez une des propositions suivantes comme mesure lue sur l'ampèremètre $A_2$ : 32,4~mA, 8,4~mA ou 16,2~mA.

\paragraph{c.} Redessinez le circuit électrique en inversant la D.E.L. Quelles seront alors les intensités mesurées par les ampèremètres $A_1$ et $A_2$ ? Expliquez pourquoi.


\section*{Exercice 4}
\begin{figure}[H]
	\centering
	\begin{circuitikz}
		\foreach \abscisse in {0, 5, 10}{%
			\draw (\abscisse,4) to[battery2, voltage=american, v={~}] ++(3,0) -- ++(0,-4) ;
		} ;
		\draw (0,4) -- ++(0,-2) to[R] ++(0,-2) -- ++(1,0) node[below] {A} to[rmeter, t=A] node[below]{COM} ++(2,0) ;
		\draw (5,4) |- ++(1,-4) node[below]{A} to[rmeter, t=A] node[below]{COM} ++(2,0) ;
		\draw (5,2) to[R] ++(3,0) ;
		\draw (10,4) -- ++(0,-2) to[R] ++(0,-2) -- ++(1,0) ;
		\draw (13,0) node[below]{A} to[rmeter, t=A] node[below]{COM} ++(-2,0) ;
	\end{circuitikz}
\end{figure}

\paragraph{1.} Dans lequel des trois circuits l'ampèremètre est-il bien positionné ?

\paragraph{2.} Pour chacun des deux autres circuits expliquez ce qui ne va pas.

\section*{Exercice 5}

~ \hfill \rule{0.5\linewidth}{0.5pt} \hfill ~

\paragraph{À noter en annexe de votre cours sur une feuille simple.} {\color{red}Un calibre est la valeur maximale qui peut être mesurée. 
Le calibre doit toujours être supérieur à la valeur mesurée sinon le fusible de l'ampèremètre grille. 
Le \emph{meilleur} calibre sera le plus proche de l'intensité mesurée tout en lui restant supérieur.}

\subparagraph*{Exemple :} Pour la valeur mesurée de I~=~126~mA, et parmi les calibres ``2~$\mu$A'', ``20~$\mu$A'', ``200~$\mu$A'', ``2~mA'', ``20~mA'', ``200~mA'', ``2~A'' ou ``20~A'' :
\begin{itemize}
	\item je peux choisir les calibres : ``200~mA'', ``2~A'' ou ``20~A'' car ces valeurs sont supérieures à 126~mA ;
	\item le meilleur calibre est : ``200~mA'' car c'est le nombre le plus proche de 126~mA et que 200 > 126.
\end{itemize}

Pour les ampèremètres de la salle 407, si le calibre choisi est 2A ou 20A les bornes ``2/20A'' et ``COM'' devront être choisies, si c'est un autre calibre, alors il faudra utiliser les bornes ``mA'' et ``COM''.

~ \hfill \rule{0.5\linewidth}{0.5pt} \hfill ~

\paragraph{1.} Un élève doit mesurer plusieurs intensités sur un ampèremètre possédant les calibres et les entrées suivantes :
\begin{table}[H]
	\centering
	\rowcolors{2}{white!80!black}{white}
	\begin{tabular}{c c c c c c}
		Valeur du calibre & 2 mA & 20 mA & 200 mA & 2 A & 20 A \\
		\hline\hline
		Borne 1 à utiliser & mA & mA & mA & 2/20 A & 2/20 A \\
		Borne 2 à utiliser & COM & COM & COM & COM & COM \\
	\end{tabular}
\end{table}

Pour chaque valeur à mesurer précise le meilleur calibre à utiliser et les deux bornes de connexion pour brancher l'appareil dans le circuit.

\textbf{a.} 126 mA \hfill \textbf{b.} 0,19 A \hfill \textbf{c.} 21 mA \hfill \textbf{d.} 325 mA

Rappel : 1 A = 1 000 mA

\paragraph{2.} Comment se branche un ampèremètre dans un circuit -- quelque soit le type de circuit ?

\section*{Exercice 6}
Pour chacun des n\oe{}uds dessinés indiquez quelle est l'intensité de la branche principale et celles des branches dérivées.

\begin{figure}[H]
	\centering
	\begin{circuitikz}
		\foreach \abscisse/\noeud in {
			1/A,
			5/B,
			9/C,
			13/D
		}{%
			\node at (\abscisse,2.5) {n\oe{}ud \noeud} ;
			\draw[dashed, blue!40!red] (\abscisse,1) circle (1.2) ;
		} ;
		% nœud A
		\draw (-0.2,1) to[short, i=$I_1$, -*] (1,1) ;
		\draw (1,1) to[short, i=$I_2$, *-] (2,2) ;
		\draw (1,1) to[short, i=$I_3$, *-] (2,0) ;
		% nœud B
		\draw (5,2.2) to[short, i=$I_1$, -*] (5,1) ;
		\draw (5,-0.2) to[short, i=$I_2$, -*] (5,1) ;
		\draw (5,1) to[short, i=$I_3$, *-] (3.5,1) ;
		% nœud C
		\draw (9,1) to[short, i=$I_1$, *-] (9,2.2) ;
		\draw (9,1) to[short, i=$I_2$, *-] (10.2,1) ;
		\draw (9,-0.2) to[short, i=$I_3$, -*] (9,1) ;
		% nœud D
		\draw (12,2) to[short, i=$I_1$, -*] (13,1) ;
		\draw (14.2,1) to[short, i=$I_2$, -*] (13,1) ;
		\draw (13,1) to[short, i=$I_3$, *-] (12,0) ;
	\end{circuitikz}
\end{figure}

\section*{Exercice 7}
\begin{figure}[H]
	\centering
	\begin{circuitikz}
		\draw (0,0) to[lamp,*-*] ++(2,0) to[rmeter, t=$A_3$, *-*] ++(2,0) -- ++(0,4) to[lamp, *-*] ++(-2,0) to[rmeter,t=$A_1$, *-*] ++(-2,0) -- ++(0,-4) ;
		\draw (0,2) to[rmeter, t=$A_2$, *-*] ++(2,0) to[battery2, *-*, voltage=american, v={~}] ++(2,0) ;
	\end{circuitikz}
\end{figure}

\paragraph{Données :} L'ampèremètre $A_1$ mesure l'intensité $I_1$ qui vaut 0,18~A, l'ampèremètre $A_2$ mesure l'intensité $I_2$ et l'ampèremètre $A_3$ mesure l'intensité $I_3$ qui vaut 0,38~A.

\paragraph{1.} Ajoute à côté du n\oe{}ud B les intensités $I_1$, $I_2$ et $I_3$ avec les flèches indiquant le sens des courants.

\paragraph{2.} quelle est la loi à utiliser dans ce circuit ? Que dit-elle ?

\paragraph{3.} Calculez $I_2$ et justifiez en écrivant les calculs effectués.

\section*{Exercice 8}
\begin{figure}[H]
	\centering
	\begin{circuitikz}
		\draw (0,3) to[battery2, *-*, i=$\mathrm{I_1}$, v=4.5V, voltage=american] ++(0,-3) ;
		\draw (0,3) -- ++(2,0) to[R, *-*,  i=$\mathrm{I_2}$] ++(0,-3) -- ++(-2,0) ;
		\draw (2,3) -- ++(2,0) to[lamp, *-*, i=$\mathrm{I_3}$] ++(0,-3) -- ++(-2,0) ;
	\end{circuitikz}
\end{figure}

\paragraph{1.} Redessinez le circuit électrique en ajoutant un ampèremètre aux endroits nécessaires pour mesurer $\mathrm{I_1}$ et $\mathrm{I_3}$.

\paragraph{2.} Quelle est la loi à utiliser dans ce circuit ? (Au n\oe{}ud B par exemple) Que dit-cette loi ?

\paragraph{3.} Calculez $\mathrm{I_2}$ et justifiez par les calculs.

\section*{Exercice 9 : guirlande de noël}
Les guirlandes de noël ont évolué depuis de nombreuses années, passant d'un branchement en série à un branchement en dérivation afin d'éviter tout risque d'accident électrique.
Les deux circuits qui suivent montrent un exemple de guirlande ancienne et un exemple de guirlande moderne.

Pour que chaque lampe fonctionne, il faut une intensité du courant électrique égale à 0,1~A.

\begin{figure}[H]
	\centering
	\begin{circuitikz}
		\foreach \abscisse/\labelle in {0/1, 9/2}{%
			\draw (\abscisse, 3) to[vsource, *-*, voltage=american, v={~}, i=$I_\labelle$] ++(0,-3) ;
		} ;
		\foreach \abscisse in {0, 1.5, 3, 4.5, 6}{%
			\draw (\abscisse, 3) to[lamp] ++(1.5,0) ;
			\draw (\abscisse, 0) to[lamp] ++(1.5,0) ;
		} ;
		\foreach \abscisse in {7.5, 10.5, 12, 13.5, 15}{%
			\draw (\abscisse,3) to[lamp] ++(0,-3) ;
		} ;
		\foreach \ordonnee in {0,3}{%
			\draw (9,\ordonnee) -- ++(6,0) ;
		} ;
	\end{circuitikz}
\end{figure}

\paragraph{1.} Calculez en justifiant l'intensité $I_1$.

\paragraph{2.} Calculez en justifiant l'intensité $I_2$.

\section*{Exercice 10 : Simulation de circuits}
\subsection*{Circuit en dérivation}
Rendez vous sur le site suivant (le lien est actif dans ce pdf) :\newline
\url{https://go-lab.gw.utwente.nl/production/electricalCircuitLab/build/circuitLab.html?preview}\newline
construisez le circuit suivant :
\begin{figure}[H]
	\centering
	\includegraphics[width=0.8500\textwidth]{circuit-deriv-utwente.png}
\end{figure}
et suivez les consignes.
\paragraph{1.} De gauche à droite les lampes s'appelleront L1, L2, L3 et L4.
Pour chaque lampe, attrapez l'ampèremètre (le A entouré de bleu dans la colonne de droite) et mesurez l'intensité de chaque lampe.
Si vous avez manipulé convenablement, la valeur affichée à droite dans la case d'où a été attrapé l'ampèremètre doit indiquer 90~mA

\paragraph{2.} En utilisant la loi des n\oe{}uds, calculez l'intensité fournie par le générateur.

\paragraph{3.} Déplacez l'ampèremètre sur un point de la branche principale et vérifiez votre réponse \textbf{2.}. Les valeurs sont-elles identiques ?

\subsection*{circuit en série}
Rendez vous sur le même site et construisez le circuit suivant :
\begin{figure}[H]
	\centering
	\includegraphics[width=0.8500\textwidth]{circuit-serie-utwente.png}
\end{figure}
et suivez les consignes.
\paragraph{1.} Du plus vers le moins les lampes s'appelleront L1, L2, L3, L4 et L5.
Pour chaque lampe, attrapez l'ampèremètre (le A entouré de bleu dans la colonne de droite) et mesurez l'intensité de chaque lampe.
Si vous avez manipulé convenablement, la valeur affichée à droite dans la case d'où a été attrapé l'ampèremètre doit indiquer 90~mA

\paragraph{2.} En utilisant la loi d'unicité, calculez l'intensité fournie par le générateur.

\paragraph{3.} Déplacez l'ampèremètre près de la pile vérifiez votre réponse \textbf{2.}. Les valeurs sont-elles identiques ?

\end{document}