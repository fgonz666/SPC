\documentclass[12pt,a4paper]{article}
\usepackage[utf8]{inputenc}
\usepackage[french]{babel}
\usepackage[T1]{fontenc}
\usepackage{amsmath}
\usepackage{xcolor}
\usepackage{amsfonts}
\usepackage{amssymb}
\usepackage{graphicx}
\usepackage{lmodern}
\usepackage[left=2cm,right=2cm,top=2cm,bottom=2cm]{geometry}
\author{F.S.G.}
\date{Création le 16 Décembre 2021\\Compilation du \today .}
\title{Exercices supplémentaires sur les énergies cinétique et potentielle.}
\begin{document}
\maketitle

\begin{abstract}
	Cette série d'exercices est tirée soit de recherches personnelles soit d'autres manuels plus anciens afin de vous permettre de maîtriser les notions d'énergie cinétique et d'énergie potentielle.
	
	Je rappelle que d'après le programme, bien que vous deviez connaître les deux notions seule la formule -- relation mathématique -- de l'énergie cinétique est à connaître, celle de l'énergie potentielle vous étant donnée à chaque fois.
	
	Par contre il est requis de savoir utiliser l'une ou l'autre de ces relations dans tous les sens, \emph{i.e.}\footnote{i.e. : \emph{id est} = C'est-à-dire en latin.} pour calculer la valeur de n'importe lequel des symboles présents.
\end{abstract}

\section*{Rappel des formules.}
\addcontentsline{toc}{section}{Rappel des formules.}
\subsection*{Énergie cinétique}
\addcontentsline{toc}{subsection}{Énergie cinétique}
Un objet de masse $m$ (en kilogramme) animé d'une vitesse $v$ (en mètre-par-seconde) possède une énergie cinétique $E_C$ (en Joule) calculée par la relation :
\[
	E_C = \dfrac{1}{2} \times m \times v^2 = \dfrac{m \times v^2}{2}
\]

\subsection*{Énergie potentielle mécanique / Énergie de position}
\addcontentsline{toc}{subsection}{Énergie cinétique}
Un objet de masse $m$ (en kilogramme) positionné à une hauteur $h$ (en mètre) sur une planète où une intensité de pesanteur $g$ (en J/kg/m) existe possède une énergie potentielle $E_P$ (en Joule) donnée par la relation :
\[
	E_P = m \times g \times h
\]

\subsection*{Énergie mécanique}
Dans certaines classes cette notion a été écrite dans la conclusion finale, mais pas dans d'autres faute de temps, aussi je vous demande de compléter la conclusion en ajoutant ce qui suit :
{\color{red} \textbf{On appelle ``Énergie mécanique'' notée $E_M$, dont l'unité est le Joule ( J ) l'addition de l'énergie cinétique $E_C$} et de l'énergie potentielle de position $E_P$ pour l'objet étudié :\newline
\[
	E_M = E_C + E_P
\]
}

\section*{Exercice 1 : L'énergie d'un véhicule}
\addcontentsline{toc}{section}{Exercice 1 : L'énergie d'un véhicule.}
\paragraph{1.} Un camion et sa cargaison, le tout cumulant une masse totale (m) de 40,000~t se déplace à une vitesse de 90~kmn/h (25~m/s). Calculez en Joule son énergie cinétique $E_{C_C}$ en justifiant correctement le résultat.

\paragraph{2.} Une voiture et ses passagers formant une masse totale (m) de 1~400 kg roule aussi à une vitesse de 90~km/h (25~m/s). Calculez son énergie cinétique $E_{C_V}$ et justifiez aussi le résultat.

\paragraph{3.} Lequel des deux véhicules possède le plus d'énergie cinétique ? Pourquoi ?

\paragraph{Aide :} 1~t~=~1~000~kg

\section*{Exercice 2 : Gouttières}
\addcontentsline{toc}{section}{Exercice 2 : Gouttières}
De plus en plus les gouttières en zinc (métal) sont remplacées par des gouttières en PVC (Polychlorure de Vinyle) une matière plastique moins dense que le métal zinc. Une maison à 3 étages envisage pour des raisons de sécurité de remplacer une gouttière en zinc (1) située à 7,5 m de haut, vieillissante et abîmée par une autre neuve (2) en P.V.C. Les gouttières (1) et (2) ont exactement le même volume et donc la même taille.

\paragraph{1.} La gouttière (1) en zinc pèse 5,30~kg. Calculez son énergie potentielle $E_{P_1}$ à la hauteur de 7,5~m de hauteur.

\paragraph{2.} En supposant que toute l'énergie potentielle devient de l'énergie cinétique, calculez quelle serait sa vitesse $v_1$ si la gouttière tombait au sol.

\paragraph{3.} La gouttière (2) en P.V.C pèse 102~g. Calculez son énergie potentielle $E_{P_2}$ à la même hauteur que la gouttière (1).

\paragraph{4.} En supposant une chute équivalente à celle de la question \textbf{2.} calculez la vitesse $v_2$ de cette gouttière au sol.

\paragraph{5.} Expliquez pourquoi il est moins dangereux d'avoir une gouttière en P.V.C. suspendue en hauteur plutôt qu'une gouttière en zinc.

\paragraph{Aides :} \hfill 1~kg~=~1~000~g \hfill g~=~10~J/kg/m \hfill ~

\section*{Exercice 3 : Houston ? On a un problème.}
\addcontentsline{toc}{section}{Exercice 3 : Houston ? On a un problème.}
Le 13 avril 1970 à 13h13 minutes décollait de Cape Canaveral (Floride, USA) la fusée ``Saturn V'' pour la mission ``Apollo XIII''. Au bout de deux jours de voyage qui devait amener cette mission à se poser sur la Lune, une explosion dans un réservoir d'oxygène transforme cette mission qui devait être tranquille en mission de sauvetage spatial (la première et la seule réussie) qui a permis de ramener sur Terre les trois astronautes.
Au bout de près de 8 jours de mission, après moult péripéties, le vaisseau spatial et son équipage, d'une masse supposée constante jusqu'à l'atterrissage de 6~000~kg (6~t) arrive à la limite de l'atmosphère terrestre avec une vitesse de 10~500~m/s.

\paragraph{1.} Calculez l'énergie cinétique $E_{C_1}$ de ce vaisseau à cet instant.

\paragraph{2.} Le vaisseau étant à une hauteur de 100~km, et l'intensité de pesanteur étant supposée égale à 10~J/kg/m, calculez l'énergie potentielle du vaisseau $E_{P_1}$.

\paragraph{3.} Lorsque le vaisseau passe la première partie de l'atmosphère il est ralenti afin que sa vitesse de chute ne soit plus que de 30~m/s. Calculez alors sa nouvelle énergie $E_{C_2}$ (on supposera la masse constante entre les deux).

\paragraph{4.} À 3~m du sol, des réacteurs produisent une poussée verticale dirigée vers le haut pour stopper le vaisseau. Vu l'inertie du véhicule on va considérer qu'à 1~m du sol, le vaisseau est totalement arrêté, ne reste plus que l'énergie potentielle $E_{P_3}$. Calculez-là en supposant que ``g'' est le même qu'à la question \textbf{2}.

\paragraph{5.} Suite à cela, le vaisseau tombe et amerrit. En supposant que toute l'énergie potentielle est transformée en énergie cinétique, quelle est l'énergie ($E_{C_3}$) que va devoir supporter le vaisseau et les astronautes lors du contact avec la surface de l'eau ?

\paragraph{6.} Quelle sera la vitesse ($v_3$) en m/s du vaisseau au contact de la mer lorsqu'il va brutalement perdre l'énergie $E_{C_3}$ ? Je vous conseille cependant de convertir la vitesse en km/h pour vous représenter au mieux cette vitesse.

\paragraph*{Aides :} \hfill 1~km~=~1~000~m \hfill g~=~10~J/kg/m \hfill ~ 1~m/s~=~3,6~km/h \hfill ~

\section*{Exercice 4 : Handball.}
\addcontentsline{toc}{section}{Exercice 4 : Handball.}
Lors du match de demi-finale de handball de la coupe d'Europe des nations, le joueur Melvyn Richardson a été enregistré lors de l'un de ses tirs en extension. La vitesse du ballon a été mesurée à v~=~133,2~km/h, son énergie calculée à 308,025~J très précisément.

Calculez la masse de ce ballon de handball en gramme.

\paragraph{Aide :} \hfill 1~kg~=~1~000~g. \hfill 1~m/s~=~3,6~km/h. \hfill ~

\section*{Exercice 5 : Un accident de voiture.}
\addcontentsline{toc}{section}{Exercice 5 : Un accident de voiture.}
Le conducteur provoque un accident en perdant le contrôle de son véhicule dans un lieu où la vitesse était limitée à 45~km/h. 
L'expert en assurance dépêché sur place pour établir les responsabilités calcule que l'énergie libérée lors du choc était de 210~000~J pour une masse totale roulante -- véhicule, carburant, occupants, bagages -- de 1~500~kg. 
L'assureur du conducteur paiera les réparations si le conducteur respectait le code de la route, donc la vitesse limite du lieu.

Qui, du conducteur ou de l'assurance paiera au final les réparations ? Justifiez votre réponse.

\paragraph*{Aide :} 1~m/s~=~3,6~km/h

%\section*{Exercice 5 : }
%\addcontentsline{toc}{section}{Exercice }
\vfill
%\newpage
\renewcommand{\baselinestretch}{1}
\setlength{\parskip}{0em}
\tableofcontents
\end{document}